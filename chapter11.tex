\newChapter 工具:微分恒等式|7
\exer 正弦函数的二倍角公式是$\sin(2x) = 2\sin(x)\cos(x)$. 利用这一点求出余弦函数 的二倍角公式.\par
\begin{align*}
\cos^2(2x) & = 1-\sin^2(2x) = 1 - 4\sin^2(x)\cos^2(x) \\
& = 1 - 4\sin^2(x)(1-\sin^2(x)) \\
& = 1 - 4\sin^2(x)+4\sin^4(x) \\
& = (1 - 2\sin^2(x))^2 \\
\cos(2x)&=1-2\sin^2(x)
\end{align*}

\exer 找出并解释下列表述中的错误:给定方程$ax^2 + bx + c = 0$, 式子两端同时对$x$求两次微分, 从而得到$2a = 0$, 因此 $a = 0$. 剩下的项 $bx + c = 0 $对 $x$ 求一次微分, 则有 $b = 0$. 因此, 对于所有的 $x$, $c$ 也等于 0.\par
Answer: 以上推导过程假设了对任意$x$方程都成立,所以得到的结论是$a=b=c=0$。

\exer 证明:对于任意一个多项式$a_0 + a_1x + a_2x_2 +\cdots+ a_nx_n$, 如果该式恒等于 0, 那么$a_0 =a_1 =a_2 =\cdots=a_n =0$.\par
Answer: 参考上一题。

\exer 在本章开头, 我们把二项式定理中的恒等式当作微分对象. 请证明二项式定理.\par
略

\exer 推导出关于$(x_1 + x_2 +\cdots + x_m)^n$的公式, 并给出证明.
\[(x_1 + x_2 +\cdots + x_m)^n=\sum_{k_1+k_2+\cdots+k_m=n}\frac{n!}{k_1!k_2!\cdots k_m!}x_1^{k_1}x_2^{k_2}\cdots x_m^{k_m}\]
提示:使用组合证明法。

\exer $\sin x$在0点附近的泰勒级数是
\[\sin x =\sum_{n=1}^\infty(-1)^{n-1}\cdot\frac{x^{2n-1}}{(2n-1)!}\]
利用微分恒等式求出$\cos x$在 0 点附近的泰勒展开式.\par
Answer: 两边求导数
\begin{align*}
\cos x &= \frac{\mathrm d}{\mathrm d x}\sum_{n=1}^\infty(-1)^{n-1}\cdot\frac{x^{2n-1}}{(2n-1)!} \\
&=\sum_{n=1}^\infty(-1)^{n-1}\frac{\mathrm d \frac{x^{2n-1}}{(2n-1)!} }{\mathrm d x} \\
&=\sum_{n=1}^\infty(-1)^{n-1}(2n-1)\cdot\frac{x^{2n-2}}{(2n-1)!} \\
&=\sum_{n=1}^\infty(-1)^{n-1}\cdot\frac{x^{2n-2}}{(2n-2)!} \\
\end{align*}

\exer 不明确地给出泰勒级数的具体表达式,只利用$\mathrm{e}^x$的性质证明:$\mathrm{e}^x$的泰勒级数是$\sum\limits_{n=0}^\infty x^n/n!$\par
Answer:$\mathrm{e}^x$的性质是导数等于其本身
\begin{align*}
&\frac{\mathrm d}{\mathrm dx} \sum_{n=0}^\infty x^n/n!\\
=&\sum_{n=1}^\infty nx^{n-1}/n! \\
=&\sum_{n=1}^\infty x^{n-1}/(n-1)! \\
=&\sum_{n=0}^\infty x^{n}/n! \\
\end{align*}

\exer 利用泰勒级数证明: $\sum_{n=1}^\infty(-1)^{n-1} /n = \log 2$.\par
Answer: 将$x=-1$代入下式得证
\[\log(1-x)=-\sum_{n=1}^\infty x^n /n\]