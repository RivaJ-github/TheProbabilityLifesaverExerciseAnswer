\newChapter 工具:卷积和变量替换|8
\exer 给出随机变量$X$和$Y$的一个例子,使得它们的概率密度函数$f$和$g$满足:$X+Y$的概率密度函数不是$(f*g)(z)$.\par
Answer:卷积要求随机变量,例如令$Y=-X$,则$Z=X+Y=0$

% \exer 考虑一颗有$n$面的均匀骰子. 假设每次抛掷骰子都是独立的. 分别求出抛掷 2 次、3 次和 4 次后, 所得到数字之和的概率密度函数.\par
% Answer:首先抛出1次的概率密度函数
% \[f_1(k)=\begin{cases}
%     \frac{1}{n} & 1 \le k \le n \\
%     0 &\text{其他值}
% \end{cases}\]
% 抛出2次的概率密度函数
% \begin{align*}
% f_2(z)&=(f_1*f_1)(z) \\
% &=\sum_k f_1(k)f_1(z-k) \\
% &=\begin{cases}
%   0  & z \le 1 \text{或} z \ge 2n+1  \\
%   \sum\limits_{k=1}^{z-1}\frac{1}{n^2} = \frac{z-1}{n^2} & 2 \le z \le n + 1 \\
%   \sum\limits_{k=z-n}^{n}\frac{1}{n^2} = \frac{2n-z+1}{n^2} & n+1 \le z \le 2n \\
% \end{cases}
% \end{align*}
% 抛出3次的概率密度函数
% \begin{align*}
% f_3(z)&=(f_2*f_1)(z) \\
% &=\sum_k f_2(k)f_1(z-k) \\
% &=\begin{cases}
%   \sum\limits_{k=z-n}^{z-1}\frac{k-1}{n^2}\cdot\frac{1}{n} =\frac{2z-n-3}{2n^2}  & 2 + n \le z \le 2n + 1 \\
%   0  & \text{其他值} \\
% \end{cases}
% \end{align*}