\newChapter 计数\uppercase\expandafter{\romannumeral 3}:高等组合学|5
\exer 在 6.1.1 节考察了关于随机播放机和 6 张 CD 的几个问题 . 我们看到 , 当 6 张CD 都被放入时 , 如果播放了 10 首歌曲 , 那么播放机选择 CD 的方法数几乎是“不允许连续播放同一张 CD 上的两首歌曲”时的 6 倍 . 如果播放了 n 首歌 , 那么当$n\to\infty$时 ,情况如何?\par
Answer:允许连续播放的方法数为$6^n$,不允许连续播放的方法数为$6\cdot5^{n-1}$,两者的比值为$1.2^{n-1}$,当$n\to\infty$时,比值趋于无穷大。

\exer 继续上面这个关于 CD 的练习 . 如果我们播放了 6 张 CD 中的 10 首歌 , 而且每次选歌时 , 每一张 CD 都等可能地被选中 , 那么至少连续听到同一张 CD 中两首歌的概率是多少?\par
\[1-\frac{6\cdot5^9}{6^{10}}\approx 86.6\%\]

\exer 在 7 场系列赛中 , 有两支球队进行比赛 , 首先赢得 4 场比赛的队伍获胜 . 在这一系列比赛中 , 如果没有哪支球队连续赢得 2 场比赛 , 那么不同的获胜方式有多少种?\par
Answer:第一场胜负的可能性有2种,后续的胜负可能性都只有1种。所以答案为
\[2\cdot1\cdot1\cdot1\cdot1\cdot1\cdot1=2\]

\exer 在 7 场系列赛中 ( 参见上一题 ), 假设每个队在每场比赛中都有 50\% 的概率获胜 . 那么 , 在进行了 4 场、 5 场、 6 场和 7 场比赛后 , 恰好分出胜负的概率分别是多少?对于最有可能发生的情况 , 你是否感到惊讶?\par
Answer1: 7场,则前6场为3比3,概率为$2*C_6^3/2^6*1/2=5/16$;\par
6场,则前5场为3比2,且第6场领先方获胜,概率为$2*C_5^3/2^5*1/2=5/16$;\par
5场,则前4场为3比1,且第5场领先方获胜,概率为$2*C_4^3/2^4*1/2=1/4$;\par
4场,则前3场为3比0,且第4场领先方获胜,概率为$2*C_3^3/2^3*1/2=1/8$;\par
说明:以5场比赛决出胜负为例,首先选择获胜队伍,有2种可能性;接着前4场的结果总共有$2^4$种结果,而满足条件的必须是选择的队伍获胜其中3场,有$C_4^3$种情况;最后第5场选择的队伍获胜的概率为1/2.

\exer 设$n$ 是一个正整数 , 现在考察$n$副特殊的牌 , 其中第$d$副牌是由点数为$ 1,2,\cdots ,d$的牌组成的;我们从 1 到$n$中随机选出一个数 , 记作$m_1$ . 接下来 , 我们从第$m_1$副牌中随机选出一个数 , 记作$m_2$ . 然后再从第$m_2$副牌中随机选出一个数$m_3$ . 把这个过程继续下去 , 会发生什么?描述地尽可能详细 ( 在某段时间内一定会发生什么 , 以及在某段时间内可能会发生什么 ). 编写一个计算机程序 , 对$n$的各种取值进行数值模拟 , 进而帮你做出一些推测 \par
Answer: 如果从第$d$副牌中选出一个数,这个数可能是$d$(概率为$1/d$,也可能比$d$小,除非$d=1$。所以随着过程的进行,我们将抽取越来越靠前的某副牌,直到抽到第1副,最终不停地抽取第一副的那种牌。

\exer 把上一个练习推广到$2n + 1$的情形 . 对于$k \in \{n + 1,n + 2,\cdots ,2n + 1\}$, 上述游戏恰好在第$k$ 次选取后结束的概率是多少?对于最有可能的 k, 你是否感到惊讶?\par
Answer: 假设$n=2$,则第$k$次选取后结束的概率为:
\[P(2, k)=\frac{1}{2^k}\]
对于$n\ge 2$,有$1/n$的概率直接结束,其他情况下将进入一个小于等于$n$的子问题。
\begin{gather*}
    P(n, 1)=\frac{1}{n} \\
    P(n, k)=\frac{1}{n}\sum_{i=2}^n P(i, k-1)
\end{gather*}
通过上述递推式,可以利用计算机获得对于任意的$n$,恰好在第$k$次抽取后游戏结束的概率。