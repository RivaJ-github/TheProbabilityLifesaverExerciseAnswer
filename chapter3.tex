\newChapter 计数\uppercase\expandafter{\romannumeral 1}|7
\exer 许多函数和序列都是用递推法来定义的. 在斐波那契数列$\{F_n\}$ 中, 后续项被定 义为前两项之和, 前两项通常是 0 和 1 或者 1 和 2(根据问题的不同, 每种定义都有其优 点和缺点). 写出斐波那契数列的递推公式, 并利用该公式证明:对于任意的$n\ge 6$, 均有$\sqrt{2}^n \le F_n\le 2^n$.\par
Answer: 由
\begin{gather*}
F_6= 8 < 2^6 \\
\frac{F_{n+1}}{F_n}=\frac{F_{n}+F_{n-1}}{F_n} < 2 
\end{gather*}
可知$n\ge 6$时,$F_n\le 2^n$成立。\par
另一方面:
\begin{gather*}
F_6=8\ge \sqrt2^6=8 \\
F_7=13\ge \sqrt2^7\approx 11.313 \\
\end{gather*}
假设$F_k\ge\sqrt2^k$且$F_{k+1}\ge\sqrt2^{k+1}$,那么
\[F_{k+2}=F_{k+1}+F_k\ge\sqrt2^k+\sqrt2^{k+1}\ge\sqrt2^k(1+\sqrt2)>2\sqrt2^k=\sqrt2^{k+2}\]
由归纳法,对于任意$n\ge 6$,$F_n\ge \sqrt2^n$成立。

\exer 抛掷一枚均匀的硬币 60 次, 计算恰好掷出 10 次正面的概率.\par
\[\mathrm{Pr}=\frac{C_60^10}{2^{60}}=\frac{60!}{2^{60}10!50!}\approx 6.54\times 10^{-8}\]

\exer 从一副洗好的牌中抽取两张,问这两张牌之和等于21的概率是多少?假设10、J、Q和K的值都是10, A牌的值是11.\par
\[\mathrm{Pr}=\frac{C_{16}^1C_4^1}{C_52^2}=\frac{64}{1326}\approx 0.0483\]

\exer 设一副牌中共有 40 张牌 (一副标准牌有 52 张, 把点数为 7、8 和 9 的牌从一副标准牌中去掉). 从中任取 5 张牌, 问取到顺子的概率是多少?取到同花的概率呢?取 到皇家同花顺 (即从 10 开始的顺子, 并且所有牌具有相同的花色) 的概率是多少?如果 从 52 张牌中抽取, 那么这些概率会有不同吗?\par
Answer: 总共有$C_{40}^5=658008$种取法。皇家同花顺共有4种,所以概率约为$6.08\times10^{-6}$。由于总可能数减少,但是皇家同花顺的数量不同,所以概率明显提高。\par
假设7-J不算顺子,顺子的可能性有$4*(4^5-4)=4080$种,概率约为0.62\%,概率提升了近一倍。\par
同花的可能性有$(C_{10}^5-4)*4=992$种,概率约为0.151\%,概率有所减少。

\exer 在不考虑次序的前提下, 把 6 个人分成每 3 人一组, 有多少种方法?\par
Answer: $C_6^3\cdot C_3^3 / A_2^2=10$。

\exer 在不考虑次序的前提下, 把 6 个人分成每 2 人一组, 有多少种方法?\par
Answer: $C_6^2\cdot C_4^2\cdot C_2^2 / A_3^3=15$。