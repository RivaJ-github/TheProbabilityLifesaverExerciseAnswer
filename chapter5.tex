\newChapter 计数\uppercase\expandafter{\romannumeral 2}:容斥原理|5
\exer 假设在每场比赛中两队获胜的概率都是相等的, 那么在7场比赛中,至少要进行5场比赛才能决出胜负(即其中一队赢得4场比赛)的概率是多少?恰好用5场比赛决出胜负的概率是多少?\par
Answer:\par 
首先,如果前四场就决出胜负,则第一场比赛可以是任意队伍获胜,接下来三场该队必须连续获胜,所以概率为$(1/2)^3=1/8$.所以至少进行5场比赛才决出胜负的概率是$1-1/8=7/8$.\par
如果恰好5场决出胜负,说明前四场是3比1,且最后一场是领先方获胜,概率为:
\[\frac{C_2^1C_4^3C_1^1}{2^5}=1/4\]

\exer 在小于或等于100的正整数中,不能被2、3或11整除的数有多少个?\par
Answer:记$A_i$为100以内能被$i$整除的数的个数,容易得到$|A_i|=\lfloor 100/i\rfloor$,由容斥原理:
\begin{align*}
A_2\cup A_3\cup A_{11} &= A_2+A_3+A_{11} - A_2\cap A_3 - A_2\cap A_{11} - A_3\cap A_{11} + A_2\cap A_3\cap A_{11} \\
&= A_2+A_3+A_{11} - A_6 - A_{22} - A_{33} + A_{66} \\
&=50+33+9-16-4-3+1=70
\end{align*}
所以不能被2、3或11整除的数有$100 - A_2\cup A_3\cup A_{11} = 30$个。

\exer 证明:当$n\to\infty$时,从$\{1,2,3,\cdots,n\}$中随机取出的一个正整数是一个完全平方数的概率趋近于0. 更进一步, 如果把“完全平方数”替换成“素数”, 结果又如何?\par
Answer:前$n$个正整数中,有$\lfloor\sqrt n\rfloor$个完全平方数,所以
\begin{align*}
P=&\lim_{n\to\infty}\frac{\lfloor\sqrt n\rfloor}{n} \\
\le&\lim_{n\to\infty}\frac{\sqrt n}{n} \\
=&\lim_{n\to\infty}n^{-1/2}=0
\end{align*}
素数的概率也趋于0,详情查找【素数定理】相关内容。

\exer 对编号为1$\sim$10的10个不同对象进行排列.使得5号不在前2个位置上,且10号不在最后2个位置上的排列方法有多少种?\par
Answer1: 记事件$A$为5号在前2个位置上,事件$B$为10号在最后2个位置上。则,
\begin{gather*}
    |A|=|B|=C_2^1A_9^9=2\cdot9!\\
    |A\cap B|=C_2^1C_2^1A_8^8=4\cdot8! \\
\end{gather*}
而我们要求的事件为$A^c\cap B^c=(A\cup B)^c$,又
\[|A\cup B|=|A|+|B|-|A\cap B|=4\cdot9!-4\cdot8!=32\cdot8!\]
而总排列方法为$10!=90\cdot8!$,所以最终答案为$58\cdot8!=2338560$。\par
Answer2: 1、5号在最后两个位置,有$C_2^1C_8^1A_8^8=16\cdot8!$种排列。2、5号既不在前两个也不在最后两个位置,有$C_6^1C_7^1A_8^8=42\cdot8!$种排列。总计$58\cdot8!$种排列

\exer 利用容斥方法来计算, 如果拿到的5张牌中至少有一张A,那么这样的5张牌共有多少种.你要确定事件$A_i$应该代表什么. 不要用全概率法则和对立事件法则来解答(但你可以用它们来检验答案).\par
Answer: 记$A_1,A_2,A_3,A_4$分别是拿到四种花色的A的事件。则
\begin{align*}
\bigcup(A_i)&=\sum_{1\le i\le 4}A_i-\sum_{1\le i<j\le 4}A_i\cap A_j+ \sum_{1\le i<j<k\le 4}A_i\cap A_j\cap A_k - A_1\cap A_2\cap A_3\cap A_4 \\
&=4A_1-6A_1\cap A_2 + 4 A_1\cap A_2\cap A_3 - A_1\cap A_2\cap A_3 \cap A_4
\end{align*}
其中,
\begin{align*}
A_1&= C_{51}^4 = 249900 \\
A_1\cap A_2&= C_{50}^3 = 19600 \\
A_1\cap A_2\cap A_3&= C_{49}^2 = 1176 \\
A_1\cap A_2\cap A_3\cap A_4&= C_{48}^1 = 48 
\end{align*}
所以
\[\bigcup(A_i) = 4\cdot 249900 -6\cdot 19600+4\cdot1176-48=886656 \]
Verify:$C_{52}^5-C_{48}^5=886656$

\exer 容斥方法最伟大的应用之一是计算孪生素数的倒数和;具有不同处理器的计算
机给出了不同的结果, 它们之间的差异导致了英特尔奔腾处理器 bug 的发现, 这给英特尔公司带来了巨大的财务损失. 阅读与之相关的内容, 并写个短评.
\par
略

\exer 假设教室里有20名学生. 第一天, 教授打算让学生们两两握手并彼此自我介绍, 那么一共要进行多少次握手?\par
Answer: $C_{20}^2=95$次。

\exer 在上一个习题中, 我们很容易认为, 相互介绍的次数等于把 20 名学生分成 10 对 的方法数. 这是因为我们可能会认为, 把学生两两分组, 然后让每对学生彼此自我介绍, 而我们只需要考察所有可能的配对方法就行了. 沿着这种思路, 相互介绍的次数就等于$\frac{20!}{10!\cdot2^{10}} = 654\,729\,075$, 但是这个值远高于实际结果. 这个论证出现了什么错误?\par
Answer: 把 20 名学生分成 10 对 的方法的方法中,任意两人互相介绍的次数被重复了多次。

在接下来的两个习题中, 考虑带有 4 个圆形刻度盘的密码锁;只有当所有刻度盘都指向唯 一密码时, 锁才能打开. 每个刻度盘上的数字都是 0 到 9.
\exer 计算可能的密码个数.\par
Answer: $10^4=40$

\exer 如果把距离定义为, 4 个刻度盘的每个都转到密码位置所移动的单位长度之和 的最小值, 那么求出一组随机密码与真实密码之间的平均距离.\par
Answer: 不妨设位置一的密码为5,则位置一随机密码值$\{0,1,2,3,4,5,6,7,8,9\}$与真实密码值的距离分别为$\{5,4,3,2,1,0,1,2,3,4\}$,平均距离为2.5,4个刻度的总平均距离为10.